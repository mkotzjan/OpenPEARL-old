\chapter{Structure of Module Import and Export Definition}

The compiler creates an xml-file per PEARL-module containing the
\begin{itemize}
\item system part information
\item problem part specifications and declarations
\end{itemize}

\section{XML-Document Structure}
The compiler has no knowledges about the nature of a system name.
Thus the system part is translated as it is --- only syntactical errors
are recognized. 
The document has the root tag \verb|<module>|, which hat the attribute
\verb|file| with the value source file name.  
All system elements are in the xml-subtree \verb|<system>|.
All problem part elements with attribute GLOBAL are located
in the xml-subtree \verb|problem|.

\begin{XMLCode}
<?xml version="1.0" encoding="UTF-8" ?>
<module file="demo.prl">
<system>
   <username .... >
   ....
   </username>
</system>
<problem>
  <spc .../>
  <dcl .../>
  ...
</problem>
</module>
\end{XMLCode}

\section{System Part Elements}
\subsection{User Names}
If a system element defines a user name the \verb|<username>| tag is 
used. Each user name is accompanied by an system name with the tag 
\verb|<sysname>|, which may have parameters.
Maybe there is a list of associations, thus there may by a tree
of \verb|<association>| tags.
The user name tag contains the attribute \verb|line| with the value of the 
source code line number.

\subsubsection{System Name and Parameters}
\label{sec_system_names}
The system name tag contains the attribute \verb|name| containing the 
name of the system element.
The parameters are located in the \verb|<parameters>|-subtree.
The imc tool supports constants of type FIXED, CHAR and BIT.
Tghe compiler detects (???? wirklich notwendig???) the type of the parameter
from the literal and passes it as content to the corresponding
\verb|<FIXED>|-, \verb|<CHAR>|- or \verb|<BIT>|-tag.

\subsubsection{Associations}
An association may be ether a system name (with parameters) or a user name.
The  \verb|<association>|-tag contains the attribute \verb|name| with the 
given name as value. If parameters are specified for the association,
they are passed as \verb|<parameters>|-subtree in the same way as decribed 
in section \label{sec_system_names}.

There is no check in the compilation phase necessary,
whether a name is a system name
or a user supplied name.


\subsection{Configuration Item}
Configuration elements in the system part are located in the
\verb|<configuration>|-element. 
The only difference to the \verb|<username>|-tag is the absence of the 
\verb|name|-attribute. Parameters and associations apply identical.

\section{Problem Part Elements}
\subsection{Specification of System Part Elements}
The PEARL source code defined the type of a system name to be ether a DATION, 
INTERRUPT or DATION. According detected type, the compiler 
adds a xml-tag \verb|<spc>|-tag with attributes \verb|type| anf \verb|line|
containing the value \verb|"dation"|, \verb|"interrupt"| or \verb|"signal"|,
 respectively.
While interrupts and signals have no parameters, dations need more
specifications. The attributes are stores as comma separated list in the
<attributes>-subelement, the transmission data is given as value
of the <data>-tag.

\section{Example}
\begin{PEARLCode}
MODULE(demo);

SYSTEM;
  lm75: LM75('48'B4) --- I2CBus('/dev/i2c-0', 100000);

  lm75a : LM75('49'B4) --- i2cbus1;
  i2cbus1: I2CBus('/dev/i2c-1', 100000);

  sig1: FixedRangeSignal;
  int1: UnixSignal(15);

  disc: Disc('/tmp/folder1', 10);
  
  stdOut: StdOut;

  Log('EW') --- StdOut;

PROBLEM,
  SPC sig1 SIGNAL GLOBAL;
  SPC int1 INTERRUPT GLOBAL;
  SPC disc DATION OUT SYSTEM ALL GLOBAL;
  SPC stdOut DATION OUT SYSTEM ALPHIC GLOBAL;

MODEND;
\end{PEARLCode}

\begin{XMLCode}
<?xml version="1.0" encoding="UTF-8" ?>
<module file="demo.prl">
<system>
<username name="lm75" line="4">
   <sysname name="LM75">
   <parameters>
   <BIT>'48'B4</BIT>
   </parameters>
   </sysname>
   <association name="I2CBus">
      <parameters>
        <CHAR>'/dev/i2c-0'</CHAR>
        <FIXED>100000</FIXED>
      </parameters>
   </association>
</username>

<username name="lm75a" line="6">
   <sysname name="LM75">
   <parameters>
   <BIT>'49'B4</BIT>
   </parameters>
   </sysname>
   <association name="i2cbus1">
   </association>
</username>

<username name="i2cbus1" line="7">
   <sysname name="I2CBus">
      <parameters>
        <CHAR>'/dev/i2c-1'</CHAR>
        <FIXED>100000</FIXED>
      </parameters>
   </sysname>
</username>

<username name="sig1" line="9">
   <sysname name="FixedRangeSignal"/>
</username>

<username name="int1" line="10">
  <sysname name="UnixSignal">
    <parameters>
      <FIXED>15</FIXED>
    </parameters>
  </sysname>
</username>

<username name="disc" line="12">
  <sysname name="Disc">
     <parameters>
       <CHAR>'/tmp/folder1'</CHAR>
       <FIXED>10</FIXED>
     </parameters>
  </sysname>
</username>

<username name="stdOut" line="14">
  <sysname name="StdOut">
  </sysname>
</username> 

<configuration line="16">
  <sysname name="Log">
    <parameters>
     <CHAR>'EW'</CHAR>
    </parameters>
  </sysname>
  <association name="StdOut">
  </association>
</configuration>
</system>
<problem>
   <spc type="signal" name="sig1" line="19"/>
   <spc type="interrupt" name="int1" line="20" />
   <spc type="dation" name="disc" line="21">
      <attributes> OUT,SYSTEM </attributes>
      <data>ALL</data>
   </spc>
   <spc type="dation" name="stdOut" line="22">
      <attributes> OUT, SYSTEM </attributes>
      <data>ALPHIC</data>
   </spc>
</problem>
</module>
\end{XMLCode}

\section{Encoding of Compound Types}
The encoding of compound types will be defined when compound types are 
available. Maybe the representation by the corresponding  C++ struct name
as described in the runtime system documentation \cite{runtime} is
suitable.

