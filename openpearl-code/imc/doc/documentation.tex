\chapter{Documentation}

\section{Architecture}

\begin{description}
\item[Pass 1:]
The tool reads the specified xml files as DOM-trees into memory.
In case of read errors, the java default exception handling
gives information about the mistake in the input files.

\item [Pass 2:]
Scan the system part for each module and add the detected system elements
into the container \verb|SystemElements|. Forward declarations are 
recorded together with the expectation. 
Errors are created on unknown system elements, wrong number or values
of parameters. Mismatch of  associations types.
Duplicate user names are detected and reported as error.

\item[Pass 3:]
The problem part elements are checked to fit to the system definitions.
Unused defined system elements are reported as warning.

\item[Pass 4:]
The C++ source code is created.

\end{description}

The process stops after each pass, if there were errors detected.

\section{Source Code}
The source code consists of several classes, whic are documented with JavaDoc.

