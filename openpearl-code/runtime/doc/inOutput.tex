\chapter{Devices and I/O}

PEARL distinguishes between system dations and user dations.
All examples in the language report shown that the statements TAKE and SEND
operate directly on system dations like a digital i/o.
The operations READ, WRITE, GET and PUT work only on user dations.
User dations are created upon a system dation.

System part and problem part may be compiled separatelly. Therefore 
no information from the system part may be used to compile the problem part.

\section{Language Problems}
The language definition lacks some details in the region of the
interface between system part and problem part as well as the 
inter module specification declaration interface.
Any incompatibilities in the interface specifications should result in
errors during the build process. Detections during runtime should be avoided.

\begin{itemize}
\item There seems to be no strict signature to distinguish
 in the {\em SPC} statement between system dation and user dation.
\item BASIC dations do not specify the type of transferred data
\item the global attribute may be omitted in the specification statement
  if the system device is defined in the same module
\item it is not clearified in the language specification, which attributes
   are required in the specification of a imported user dation.
\item a new optional class attribute is introdued {\em SYSTEM}, which
   must be set for system dations
\item the class attribute {\em BASIC} must be combined with a data type
    like {\em ALL} or {\em any other specified type}. This allows
    type checking at {\em TAKE} and {\em SEND}.
\item there should be no intermixing of system and problem part.  ---
   The specification of a system device need always a global attribute.
\item the specification of a user dation need the specification of direction
   and class attribute.
\end{itemize}

Example:
\begin{PEARLCode}
MODULE (m1);
SYSTEM;
   output: StdStream(stdout);
PROBLEM;
   SPC output  DATION OUT SYSTEM ALPHIC GLOBAL;
   DCL console DATION OUT ALPHIC DIM(*,80) FORWARD STREAM NOCYCL CREATED(output) GLOBAL;
MODEND;
...
MODULE (m2);
PROBLEM;
   SPC output  DATION OUT SYSTEM ALPHIC GLOBAL;
   SPC console DATION OUT ALPHIC /* DIM(*,80) FORWARD STREAM NOCYCL CREATED(output) */ GLOBAL;
MODEND;
\end{PEARLCode}

Remarks:
\begin{itemize}
\item the dation parameters in the comment in the specification
   are not required 
\end{itemize}

\section{Not Supported Language Elements}
\begin{description}
\item[TFU] should be ignored by the compiler
\item[TFU(MAX)] should be ignored by the compiler
\item[FORBACK] is not supported yet, since tape drives are
   difficult to find
\item[R()] remote format must be treated by the compiler
\end{description}

\section{Declarations in System Part}
There is currently no rule how to compile the system part into C++.

To provide complete code examples the linux target is used here.
For details of the used system devices please check chapter \ref{x8632devices}.

\begin{PEARLCode}
MODULE (m1);
SYSTEM;
   output: StdStream(1);
PROBLEM;
MODEND;
\end{PEARLCode}

\begin{CppCode}
static pearlrt::StdStream s_output(1);
       pearlrt::Device * d_output = &s_output;
\end{CppCode};

All devices are derived from {\em Device}.
Devices for basic dations are derived from {\em SystemDationB},
all other devices are derived from {\em SystemDationNB}.
The upcast of the pointer
to the generic {\em Device*} type enables the compiler to generated 
suitable code for the problem part.

The concrete device object is set to {\em static} to avoid namespace pollution.
Only the pointer is needed.

\section{Decoration Scheme for Devices}
We need different obejcts, which concern the same user object.
The user objects are prependend with an underscore (\verb|_|).

\begin{description}
\item[d\_...] denotes the upcasted to {\em Device} 
\item[s\_...] denotes the (static defined) real system device object
\item[\_...] without other prefix)denotes the downcasted object in 
     the C++ code of the problem part
\item[dim\_...] denotes the dimension object, which is relate to the 
   user dation
\end{description}

\section{Bus Device Associations}
The update of the language definition introduced {\em BusDeviceAssociations},
which allows to interconnect system part elements.

E.g.
\begin{PEARLCode}
MODULE (m1);
SYSTEM;
   i2c: I2CBus('/dev/i2c-0',100000);  ! access to i2-c
                                      ! interface with 
                                      ! 100kHz bus speed
   temp: LM75('48'B4) --- i2c;        ! one temperatur
                                      ! sensor at
                                      ! the bus i2c
PROBLEM;
   SPC temp DATION SYSTEM BASIC FIXED(15);
MODEND;
\end{PEARLCode}

\begin{CppCode}
// system part only
static pearlrt::I2CBus s_i2c("/dev/i2c-0",100000);

static pearlrt::LM75 s_temp(&s_i2c,0x48);
       pearlrt::Device * d_temp = &s_temp;
\end{CppCode}

Depending to the characteristics of the bus there may be only one client,
or multiple clients. This is specified in the XML-description of the 
device.

The implementation provides two roles:
\begin{itemize}
\item ConnectionProvider, which represents the bus device
\item ConnectionClient, which represents the component on the bus.
\end{itemize}

The classes for clients are derived from \texttt{SystemDationB}.
The classes for providers are be derived from a class which
defines the interface for the connection. 

E.g.:
\begin{description} 
\item[ \texttt{I2CProvider} ] is common to all plattforms
   and defines the communication interface between provider and client
\item[ \texttt{I2CBus}] implements the access to the i2c-bus on a 
   specific plattform
\item[ \texttt{LM75} ] defines a system dation, which is attached to an i2c-bus.
    This element is plattform independent.
\end{description}

\paragraph{Note:} The client has access to the provider by default, since
the provider is passed as first argument in the ctor of the client.

\paragraph{Note:} The second parameter in I2CBus is used in microcontroller
   environmants. The linux systems define the transmission speed at
   another location.

\section{Common Elements and Devices for all Plattforms}
\subsection{I2C-Bus}
The class \texttt{I2CProvider} defines the interface for all i2c-bus 
implementations. Each implementation open the i2c hardware interface 
when the ctor is called. The implementation must provide aside of a
convenient ctor the following methiods:

\begin{description}
\item[readData] read the given number of bytes from the specified 
   i2c-addres
\item[writeData] send the given number of bytes to the specified
    i2c-addres
\item[rdwr] perform several subsequent read and write operations 
   without intermediate stop condition. If the concrete controller
   does not support this method, it must throw the InternalDationSignal.
\end{description}

\subsection{I2c-Device LM75}
The sensor delivers the current temperature.

\begin{tabular}{|l|p{10cm}|}
\hline
Direction & IN \\
\hline
Type & FIXED(15) \\
\hline
Remarks& The temperature is returned in tenth of degree centigrade\\
& The chip updates the temperature value after end of data transmission. 
  In full speed polling mode, the gap between two read operations is too 
  short ($\approx 40 \mu s$) to update the value.\\
\hline
\end{tabular}

\section{Specifications in Problem Part}
The system device is specified in the problem part.
Depending on the class specifier in the {\em SPC}-statement the concete 
dation type is deduced (see tab. \ref{dationTypes}).

\begin{table}[bpht]
\begin{tabular}{l|l|l|l}
  & BASIC & ALPHIC & ALL / type \\
\hline
SYSTEM & SystemDationB & SystemDationNB & SystemDationNB \\
 & DationTS & DationPG & DationRW \\
\hline
\end{tabular}
\caption{Deduced type of the dation object}
\label{dationTypes}
\end{table}


\begin{PEARLCode}
SPC output DATION OUT SYSTEM ALPHIC GLOBAL;
SPC disc DATION OUT SYSTEM ALL GLOBAL;
SPC console DATION OUT ALPHIC GLOBAL;
SPC logbook DATION OUT Fixed(15) GLOBAL;
SPC mot DATION OUT SYSTEM BASIC BIT(4) GLOBAL;
\end{PEARLCode}

\begin{CppCode}
extern pearlrt::Device * d_output;
static pearlrt::SystemDationNB _output* = 
             static_cast<pearlrt::SystemDationNB*>(d_output);

extern pearlrt::Device * d_disc;
static pearlrt::SystemDationNB _disc* = 
             static_cast<pearlrt::SystemDationNB*>(d_disc);

extern pearlrt::DationPG _console;

extern pearlrt::DationRW _logbook;

extern pearlrt::Device * d_mot;
static pearlrt::SystemDationB * _mot = 
             static_cast<pearlrt::SystemDationB*>(d_mot);
\end{CppCode}

\section{Declaration of a User Dation}
The declaration of user dation may appear inside and outside of procedures and 
tasks.
The declaration leads to an instantiation of an object. 
The class attribute of the user dation decide about the type of the object
(see tab. \ref{dationTypes}).

The different types of possible dimension specifications are mapped to
a class hierarchy for {\em DationDim}s. For details see the doxygen part 
of the runtime documentation.

\begin{PEARLCode}
DCL console DATION OUT ALPHIC DIM(*,80) FORWARD STREAM NOCYCL CREATED(output);
DCL file DATION OUT FIXED(15) DIM(*,80) FORWARD STREAM NOCYCL CREATED(output);
DCL file1 DATION OUT ALL DIM(*) FORWARD STREAM NOCYCL CREATED(output);
DCL motor DATION OUT BASIC BIT(4) CREATED(mot);
\end{PEARLCode}

\begin{CppCode} 
// 2-dimensions, first dimension is unbound
static pearlrt::DationDim2 dim_console(80);
static pearlrt::DationPG _console(_output, 
                pearlrt::Dation::OUT      |
                pearlrt::Dation::FORWARD  | 
                pearlrt::Dation::STREAM   |
                pearlrt::Dation::NOCYCL,
                &dim_console);

static pearlrt::DationDim2 dim_file(80);
static pearlrt::DationPG _file(_output, 
                pearlrt::Dation::OUT      |
                pearlrt::Dation::FORWARD  | 
                pearlrt::Dation::STREAM   |
                pearlrt::Dation::NOCYCL,
                &dim_file, sizeof(pearlrt::Fixed<15>));

static pearlrt::DationDim1 dim_file1();
static pearlrt::DationPG _file1(_output, 
                pearlrt::Dation::OUT      |
                pearlrt::Dation::FORWARD  | 
                pearlrt::Dation::STREAM   |
                pearlrt::Dation::NOCYCL,
                &dim_file, 1);

static pearlrt::DationTS _motor(_mot,
                pearlrt::Dation::OUT);
\end{CppCode}

Remarks:
\begin{itemize}
\item the dation parameters are \verb|or|ed together
\item the dimenion object is defined statically.
\item the {\em CREATED()} parameter is passsed as first parameter
\item if the dation is defined as {\em GLOBAL} the \verb|static| must
  be omitted
\item a user dation with transfer type ALL must be 1-dimensional --- may
be bounded --- and the size of a data element must be specified as 1
\item the userdations of type BASIC do not allow Topology- and 
AccessAttribut-specifications 
\end{itemize}

\subsection{DIRECT Dation}
DIRECT dations are implemented on e.g. disc files
\begin{itemize}
\item    as artificial array like structure
\item    the DIM-specifies the structure
\item    absolute positioning is allowed
\item    relative positioning is mapped on the absolute positioning
\item    STREAM allows the positioning across dimension limits
\item    STREAM allows silent reading and writing across dimension limits
 \item   CYCLIC realizes a cyclic positioning modulo the complete
        dimension in both directions
\item    CYCLIC realizes a cyclic writing in that way that a
          repositioning to the beginning of the dation is inserted
          at the end of the dation
\item    unwritten locations in the dations contain undefined values
\item    CYCLIC on unbounded dations is ridiculous
\end{itemize}

\subsection{FORWARD Dation}
FORWARD and CYCLIC is ridiculous since FORWARD can not rewind.
FORWARD requires NOCYCL by logic.

FORWARD dations are typically used on devices like stdin, 
stdout or TCP/IP-sockets.

\begin{description}
\item[PUT/GET] dations are created on e.g. stdout. 
   \begin{itemize}
   \item PUT ... BY X, SKIP and PAGE write space, new line or formfeeds
         as required by format
   \item  GET ... BY X, SKIP and PAGE discard input characters until the
         required number of characters, newlines or formfeeds were detected
   \item NOSTREAM causes 
        the specified dimension is used for error detection. 
         GET/PUT across boundary cause an error condition
   \item STREAM:
         number of dimensions control the possibility for X,SKIP and PAGE
         no boundary enforced
   \end{itemize}

\item[READ/WRITE] positioning:
   \begin{itemize}
     \item WRITE ... BY  X,SKIP and PAGE fills zero-records as required 
           by the current location
     \item READ ... BY  X,SKIP and PAGE discards input until 
           the required location is reached.
   \end{itemize}
\item[READ/WRITE + NOSTREAM] dations are implemented on e.g.
        TCP/IP socket or pipes
   \begin{itemize}
    \item The specified dimension is used for error detection
    \end{itemize}
\item[ READ/WRITE + STREAM] will  
        normalize the current location to be within the specified dimensions
        by calculating any out of bounds position by simulating line
        and page wraps. Eg:
\begin{verbatim}
DIM( 10,80)
! current location: (5,79)
WRITE a,b,c TO ...; --> new logical location (5, 81)
WRITE d TO .. BY SKIP;
\end{verbatim}
    \begin{itemize}
     \item calculate virtual normalized location: (6,1)
     \item SKIP fills 79 records with 0; new location (7,1)
     \item d will be written on location (7,1); new location is  (7,2)
     \end{itemize}
\end{description}

\section{Concurrency during I/O}
\label{ioconcurrency}
It is never stated but clearly appreciated that i/o-statements from
different tasks should not intermix within their i/o-operations.
This behavior is achieved by the {\em beginSequence} and 
{\em endSequence} clause. 
This clause locks mutex of the dation object to block concurrent
i/o-statments upon the same dation.
The task''s object reference (me-pointer) is passed with
the {\em beginSequence} call.

In case of abnormal termination due to signals in the try-block
the {\em endSequence} in the catch-clause releases the used mutex.
If a RST-value is specified in the statement the given variable is
updated, else the exception is rethrown.

Task suspending and termination within the try-block would cause an
eternal blocking of the dation. 
This behavior is not desired.
Thus the suspend and terminate will not be done only at the end of a record
{\em SKIP}) or after the completion of the i/o-statement.
This is realized internally in the runtime system.

\begin{PEARLCode}
DCL (x,y) Fixed(15);
DCL rst Fixed(31);
...
PUT x,y TO console BY F(4), RST(rst), F(5), SKIP;
\end{PEARLCode}

\begin{CppCode}
Fixed<31> rst;
Fixed<15> x
Fixed<15> y;

try {
   _console.beginSequence(me);
   _console.toF(x,(pearlrt::Fixed<31>)4);
   _console.rst(rst);
   _console.toF(y,(pearlrt::Fixed<31>)5);
   _console.toSkip((pearlrt::Fixed<31>)1);
   _console.endSequence();
} catch (Signal& s) {
   if (!_console.updateRst(&s)) {
      _console.endSequence();
      throw;
   }
   _console.endSequence();
}
\end{CppCode}

The try-catch-clause captures all PEARL-signals from the code in the 
type block. If a RST-value is specified in the format list, the method
{\em updateRst} will assign the actual signal code to the RST-variable.
If no RST-vale is specfied for this statement, the signal is propagated.
If the error is produced in the formatting of x, a signal is thrown.
If the error is produced in the formatting of y, no signal is thrown but rst
is set.

\section{Open and Close}
The super class UserDation knows the OPEN and CLOSE operations.
Both have optional parameters. This is solved by a \verb|or|ed
parameter.

\begin{PEARLCode}
DCL f FIXED(31);
...
OPEN console;
...
OPEN logbook BY RST(f), IDF('file.001') ANY;
...
CLOSE console;
...
CLOSE logbook BY CAN;
\end{PEARLCode}

\begin{CppCode}
Fixed<15> f;
...
_console.dationOpen();
...
{
   static Character<8> fileName("file.001");
   _logbook.dationOpen(Dation::IDF|Dation::ANY| Dation::RST, &fileName ,& f);
}
...
_console.dationClose();
...
_logbook.dationClose(Dation::CAN);
\end{CppCode}

Remarks:
\begin{itemize}
\item The presence of a rst-value is signaled by the \verb|or|ed parameter
   in dationOpen and dationClose
\item The filename must be passed by reference. The corresponding
   object must be defined. Eg. as \verb|static| is a new block.
\item The runtime system checks whether the dation parameters from
   user dation match the system dation's parameters. 
   If they do not match a PEARL signal is raised
\end{itemize}

\section{Positioning}
User dations with specified topology allow
 to position with POS, ADV, SKIP, LIN, COL, ...

The following positioning methods are supplied.
All methods are available for output and input. The versions for 
output have the prefix \verb|to|, the input version have the prefix \verb|from|.
To avoid redundancy only the output versions are listed below.

Please note that all parameters are expected as type
\verb|pearlrt::Fixed<31>|.

\begin{methodMapping}
POS(x) & toPos(x) & \\
POS(x,y) & toPos(x,y) & \\
POS(x,y,z) & toPos(x,y,z) & \\
SOP(x) & toSop(x) & \\
SOP(x,y) & toSop(x,y) & \\
SOP(x,y,z) & toSop(x,y,z) & \\
ADV(x) & toAdv(x) & \\
ADV(x,y) & toAdv(x,y) & \\
ADV(x,y,z) & toAdv(x,y,z) & \\
X(x) & toX(x) & \\
SKIP(x) & toSkip(x) & \\
PAGE(x) & toPage(x) & \\
LINE(x) & & {\em not supported yet} \\
COL(x) & & {\em not supported yet} \\
\end{methodMapping}

Note:
\begin{itemize}
\item {\em THIS IS SUBJECT OF DISCUSSION!}
\item a user dation with type {\em ALL} does not specify the size of 
   the record. No positioning is allowed on such a FORWARD dation. 
   The DIRECT attribute will allow to rewind the dation by POS(1),
\item as alternative one dimension is allowed and the positions are
    expected to be calculated by the application as byte offsets.
\end{itemize}   

\section{PUT/GET Operations}
PUT and GET are allowed on ALPHIC user dations. They are mapped on the
class {\em DationPG}. 
The compiler must produce a list of operations according the sequence 
in the format list.
This list of statements must be embedded in the try/catch/block described in
\ref{ioconcurrency}.

In the following table, the parameter denotes the corresponding data element.
In PUT-statments the prefix \verb|to| is set to the formatting methog,
in GET-statements the prefix \verb|from| is used.
To avoid redundancy, only PUT-formatting is listed below.

Please note that all parameters (except the value itself)
 are expected as type
\verb|pearlrt::Fixed<31>|.

\begin{methodMapping}
A & toA(x) & \\
A(w) & toA(x,w) & \\
F(w) & toF(x,w) & \\
F(w,d) & toF(x,w,d) & \\
F(w,d,s) & toF(x,w,d,s) & \\
E(w) & toE(x,w,0,1,2) & \\
E(w,d) & toE(x,w,d,1,2) & \\
E(w,d,s) & toE(x,w,d,s,2) & \\
E(w) & toE(x,w,0,1,2) & \\
E3(w) & toE(x,w,0,1,3) & \\
E(w,d) & toE(x,w,d,1,3) & \\
E(w,d,s) & toE(x,w,d,s,3) & \\
D(w) & toD(x,w) &  \\
D(w,d) & toD(x,w,d) &  \\
T(w) & toT(x,w) &  \\
T(w,d) & toT(x,w,d) &  \\
B(w) & toB1(x,w) &  \\
B1(w) & toB1(x,w) &  \\
B3(w) & toB3(x,w) & {\em not implemented yet} \\
B4(w) & toB4(x,w) &  \\
LIST &  & {\em should be treated by the compiler} \\
\end{methodMapping}

\paragraph{Note:} The corresponding method for input is named with 'from' 
instead of 'to' (e.g. fromF(..) instead of toF(...)).


\section{READ/WRITE Operations}
READ and WRITE are allowed on user dations with a specifig type or of type ALL.
They are mapped on the class {\em DationRW}. 
The compiler must produce a list of operations
\begin{enumerate}
\item perform all positioning operations first
\item perform the data transfer in sequence of the given item list
\end{enumerate}
This list of statements must be embedded in the try/catch/block described in
\ref{ioconcurrency}.

For the data transfer two operations are defined. 

\begin{description}
\item[dationRead]{get a pointer to the data storage and the size of the data}
\item[dationWrite]{get a pointer to the data storage and the size of the data}
\end{description}


\begin{PEARLCode}
DCL (x,y) Fixed(15);
DCL rst Fixed(31);
...
WRITE x,y TO dataFile BY RST(rst), POS(10), ADV(1);
\end{PEARLCode}

\begin{CppCode}
Fixed<31> _rst;
Fixed<15> _x
Fixed<15> _y;

try {
   _dataFile.beginSequence(me);
   _dataFile.rst(_rst);
   _dataFile.toPos((pearlrt::Fixed<31>)10);
   _dataFile.toAdv((pearlrt::Fixed<31>)1);
   _dataFile.DationWrite(&_x, sizeof(_x));
   _dataFile.DationWrite(&_y, sizeof(_y));
   _dataFile.endSequence();
} catch (Signal& s) {
   if (!_dataFile.updateRst(&s)) {
      _dataFile.endSequence();
      throw;
   }
   _dataFile.endSequence();
}
\end{CppCode}

The try-catch-clause captures all PEARL-signals from the code in the 
type block. If a RST-value is specified in the format list, the method

Remarks:
\begin{itemize}
\item The runtime system checks whether the operation is allowed according
    the dation type (direction, positioning)
\item The runtime system passes the raw data to the system device to
   perform the i/o-operation
\end{itemize}

\section{TAKE/SEND Operations}
TAKE and SEND are allowed on userdations of type BASIC.
 They are mapped on the
class {\em DationTS} operations. 

\begin{methodMapping}
TAKE & \verb|dationRead(void* dst, size_t size)| & \\
SEND & \verb|dationWrite(void* dst, size_t size)| & \\
\end{methodMapping}

The compiler must produce a list of operations according the sequence 
in the format and data list.
\begin{enumerate}
\item insert the rst-setting first
\item the data transfer 
\end{enumerate}
This list of statements must be embedded in the try/catch/block described in
\ref{ioconcurrency}.

\begin{PEARLCode}
DCL x Bit(4);
DCL rst Fixed(31);
...
SEND x TO motor BY RST(rst);
\end{PEARLCode}

\begin{CppCode}
BitString<4> _x
Fixed<31> _rst;

try {
   _motor.beginSequence(me);
   _motor.rst(_rst);
   _motor.dationWrite(&_x, sizeof(_x));
   _motor.endSequence();
} catch (Signal& s) {
   if (!_motor.updateRst(&s)) {
      _motor.endSequence();
      throw;
   }
   _motor.endSequence();
}
\end{CppCode}


Remarks:
\begin{itemize}
\item S and CONTROL are not supported
\item the laguage specification specifies that there
   may be only one data element in a TAKE or SEND statement
\end{itemize}

\section{Interface for User supplied Drivers for Basic Dations}
{\bf \it PRELIMINARY} For currently not supported system interfaces a C++ class must be supplied to 
provide the system device.

\subsection{System Dation Name}
Define the name and parameters for the system dation.
The new class must be derived from \verb|SystemDationB|.
The parameters will be passed to the constructor of the class.

Basic dations are used in combination with the PEARL statements TAKE and
SEND.

\subsection{Open}
A method \verb|dationOpen(int openParam=0, Character<S>* idf=0, Fixed<31>* rstValue)|
must be provided.
The method is called when the PEARL program opens the dation.
The specified open parameters are passed bitwise encoded in \verb|openParams|.
If IDF- and/or RST-value are specified they are passed as pointers, else
a null pointer is given.
The method should make shure that the dation is not opened twice and
the open parameters are ok.

\paragraph{Note:} The idf parameter is a templated type.
 Thus the dationOpen-method should be realized as a template method,
 which delegates the concrete value of the parameter to an
 internalOpen-method in an appropriate way.

\subsection{Close}
A method \verb|dationClose(int closeParams=0, Fixed<31>* rstValue)|
must be provided.
The method is called when the PEARL program closes the dation.
The specified close parameters are passed bitwise encoded in \verb|closeParams|.
If an RST-value is specified it passed as pointer, else
a null pointer is given.
The method should make shure that the dation is not closed twice and
the close parameters are ok.

\subsection{Read}
A method \verb|dationRead(void * destination, size_t size)|
must be provided. The method shall transfer \verb|size| bytes of
input data to the location specified by \verb|destination|.
The runtime system expects that this method blocks the calling thread,
if no data are available.
The method should make shure that only an opened dation may be used.

\subsection{Write}
A method \verb|dationWrite(void * source, size_t size)|
must be provided. The method shall transfer \verb|size| bytes of
output data from the location specified by \verb|source|.
The runtime system expects that this method blocks the calling thread,
until all data is transfered.
The method should make shure that only an opened dation may be used.

\subsection{Error Conditions}
The methods may throw C++ exceptions, which are defines in the signal list.
Existing signals should be reused as far as possible.

\section{Interface for User supplied Drivers for Non Basic Dations}
For currently not supported system interfaces a C++ class must be supplied to 
provide the system device.

\subsection{System Dation Name}
Define the name and parameters for the system dation.
The new class must be derived from \verb|SystemDationNB|.
The parameters will be passed to the constructor of the class.

Non basic dations are used in combination with the PEARL statements READ, WRITE
or PUT,GET. 

\subsection{Capabilities}
The input and output routines for non basic dations 
need information about the capabilities of the
real device. The method \verb|capabilities()| must return a set of
capabilities, which are provided by the device. The capabilities are
defines as \verb|enum| in the \verb|Dation|-class.

The runtime system checks whether the requested operations fit to the 
device capabilities. The capabilities of the device are defined by this
method. Multiple capabilities from the list below may be set via bitwise
or operations.

\begin{description}
\item[IN] the device may be used as pure input dation
\item[OUT] the device may be used as pure output dation
\item[INOUT] the device may be used as dation for random directions
    input and output
\item[IDF] the device {\em needs} a file name for the open-statement
\item[OLD] the device may contain persistent files,
          which are indented to be read
\item[NEW] the device allows the creation of files 
\item[ANY] the device  supports files at all; at least 
            this attribut should be set if IDF is set
\item[PRM] the device supports persistent files
\item[CAN] the device supports the removal of files
\item[DIRECT] the device supports direct positioning (on given byte adresses)
\item[FORWARD] the device allows sequential read and write operations
\item[FORBACK] {\em NOT SUPPORTED} --- the device allows relative positioning
\end{description}
Note that the capability
\begin{itemize}
\item IDF needs any of the capabilites OLD, NEW, ANY.
\item OLD needs the capability IN
\item NEW needs the capability OUT
\item FORWARD and DIRECT may be both
\end{itemize}

\subsection{Open}
A method \verb|dationOpen(const char * idfValue, int openParams)|
must be provided.
The method is called when the PEARL program opens the dation.
The current {\em openParams} must be treated according the language definition.

\subsection{Close}
A method \verb|dationClose(int closeParams)|
must be provided.
The method is called when the PEARL program closes the dation.
The current {\em closeParams} must be treated according the language 
definition.

\subsection{Read}
A method \verb|dationRead(void * destination, size_t size)|
must be provided. The method shall transfer \verb|size| bytes of
input data to the location specified by \verb|destination|.
The runtime system expects that this method blocks the calling thread,
if no data are available.

\subsection{Write}
A method \verb|dationWrite(void * source, size_t size)|
must be provided. The method shall transfer \verb|size| bytes of
output data from the location specified by \verb|source|.
The runtime system expects that this method blocks the calling thread,
until all data is transfered.

\subsection{Error Conditions}
The methods may throw C++ exceptions, which are defines in the signal list.
Existing signals should be reused as far as possible.

