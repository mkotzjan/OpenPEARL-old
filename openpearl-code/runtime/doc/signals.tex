\chapter{Signals}

The signal definitions are stored in the OpenOffice document {\em Signals.ods}
for all plattforms.
As shown in fig. \ref{signals_ods} the perl script
{\em GenerateSignalDefinitions.pl} creates two files
{\em Signals.hh} and {\em Signals.hcc} containing the class definitions 
and static objects declarations. 

\begin{figure}[bpht]
\begin{center}
\includegraphics[width=10cm]{signals_ods.jpg}
\end{center}
\caption{Signal definitions by Spreadsheet}
\label{signals_ods}
\end{figure}

The generation is triggered by the plattform specific {\em Makefile}.
The script takes the desired target specifier(s) as command line
parameters.

\section{Signal Definition}
The first sheet in {\em Signals.ods} contains the administrativ part.

\subsection{Grouping}
The signals are stored in groups. The definition of a group requires
an entry in the list of groups, which resides in the first sheet {\em GROUPS}
starting at line 3\footnote{This location must not be changed!}.
Each group entry consists of
\begin{description}
\item[Group name] which must have an identical name to the spreadsheet page.
   The sequence of spreadsheet pages must be identical to the sequence of
   group entries in this section.
\item[Group Number] defines the number region of the group.
    The {\em Group Number} is multiplied internally by 100.
    There may be 99 signals within each group.
\end{description}
The scripts checks for the matching of the group names and for multiple
usage of a group number.
The list of groups is terminated by an emtpy line.

\subsection{Groups}
Each group element contains the following entries:
\begin{description}
\item[GROUP INTERNAL ID] is the offset to the {\em Starting Number} of the 
   group. The id must be unique for all signals (not checked) to allow
   proper signal treatment by the runtime system.
   The {\em group internal id} 0 denotes the signal matching
   all signals defined in the group.
   The resulting signal {\em id} is created by the calculation of
   $GroupNumber \cdot 100 + GroupInternalId$
 
\item[ExceptionClass] defines the name of the C++ class, which shall be created.
   These names must by unique over all groups. This is also the name of
   the signal in the system part.
\item[ParentClass], which is used to build a C++ class hierarchy of the
    signals (C++ exceptions).
\item[System] specifies which plattform needs this signal. 
   The valid entries are defined by a drop down list. Please update the lists 
   on all sheets when a new plattform is added.
\item[Message] is the text which is sent to the standard output, when this
    signal was detected by the default signal handler.
\item[Description] contains the description of the reason for this signal
   for the external plattform signal manual. TeX syntax is allowed.
\end{description}

The scripts captures all entries of all group sheet with a matching
{\em System} entry.

\section{Decoration Scheme}
\begin{description}
\item[xxx] is the user supplied identifier for a signal
\item[\_xxx] is the (private) identifier with the concrete signal object
\item[generalized\_xxx] is the upcasted (public) identifier 
    of the concrete signal
\item[index\_xxx] is the enumeration item for the signal index within
   one procedure ore task

\item[lll] is the user defined identifer for a label 
\item[label\_lll] is the label in C++
\item[index\_lll] is the idenx of the label in the array for the computed goto

\end{description}

\section{Usage in System Part (is subject of change)}
A plattform specific signal is mapped to a user defined identifer in the 
system part. 

\begin{PEARLCode}
CHAR_OVERFLOW: CharacterTooLongSignal;
\end{PEARLCode}

\begin{CppCode}
static pearlrt::CharacterTooLongSignal _CHAR_OVERFLOW;
pearlrt::Signal * generalized_CHAR_OVERFLOW = &_CHAR_OVERFLOW;
\end{CppCode}

There is no possibility to define a signal upon its number, since this
produces lots of problems with the {\em static initialisation desaster}.
The identification of a signal at run time is done upon the signal id
which is referenced as {\em RST-value}.


\section{C++ Code for Signal Handler}
PEARL defines that a signal handler may be scheduled not inside 
a BEGIN/END-clause. Thus is is allowed inside of IF/THEN/ELSE-blocks
or on TASK or PROC level.

Possible reaction is a block of statements, which is executed and the
PROC is exited (the TASK teriminated) if the signal handler does not
contain a {\em GOTO-statement} defining the continuation point.

These code structure requires that only labels defined on task or proc level
may be used for this goto.

\begin{PEARLCode}
SYSTEM;
   overfl: FixedOverFlowSignal;    
   div0:   FixedDivideByZeroSignal;
   arith:  ArithmeticSignal;

 PROBLEM;
   SPC overfl SIGNAL;
   SPC div0   SIGNAL;
   SPC arith  SIGNAL;
   X: PROC (p FIXED) 
      DCL k FIXED(15);

      k := 2;

restart: 
      ! signal action #1
      ON overfl BEGIN
         PUT 'PROC X: Arithmetic error (returning)' TO console BY A, SKIP;
      END;

      ! signal action #2
      IF ( p == 1) THEN
         ON overfl BEGIN
            PUT 'PROC X: Overflow occured (terminating)' TO console BY A, SKIP;
            TERMINATE;     
         END;
      FIN;
      
      IF ( p > 5) THEN
      ! signal action #3
        ON overfl BEGIN
           PUT 'PROC X: Overflow occured (returning)' TO console BY A, SKIP;
        END;
      FIN;

      ! signal action #4
      ON div0 BEGIN 
          PUT 'Divide by zero (restarting)' TO console BY A, SKIP;
          IF ( p ==  6) THEN
             goto exit;
          ELSE
             p := 6;
             goto restart;
          FIN;
      END;
      
      IF p == 11 THEN
          k := 10 / (k-k);  ! force 'div by 0'
      FIN;

      FOR i TO 100 REPEAT
        k := k * k;
      END;
exit:
  END;
\end{PEARLCode}
There are four signal handler defined in the procedure.
Since there are signals handler defined in the procedure,
the body is placed in a try-catch-block.
The catch block checks, whether the current signal matches 
on of the scheduled signal actions.
If {\em goto} is used inside a signal handler the procedure entry 
gets a computed goto, which is provides by GCC extension 
{\em labels as values}.

\begin{CppCode}
// system
static pearlrt::FixedRangeSignal _overfl;
pearlrt::Signal *generalized_overfl = &_overfl;


static pearlrt::FixedDivideByZeroSignal _div0;
pearlrt::Signal *generalized_div0 = &_div0;

static pearlrt::ArithmeticSignal _arith;
pearlrt::Signal *generalized_arith = &_arith;

// problem
extern pearlrt::Signal *generalized_overfl;
extern pearlrt::Signal *generalized_div0;
extern pearlrt::Signal *generalized_arith;

void _x(pearlrt::Task * me, pearlrt::Fixed<15> p) {
   pearlrt::Fixed<15> _k;

   //[ generated from ON statements in PROC
   static pearlrt::ScheduleSignalAction sigActions[] = {
		pearlrt::ScheduleSignalAction(generalized_arith),
		pearlrt::ScheduleSignalAction(generalized_overfl),
		pearlrt::ScheduleSignalAction(generalized_div0)};

   const size_t NBROFSIGACTIONS = 
		sizeof(sigActions)/sizeof(sigActions[0]);

   enum signalIndex{index_arith, index_overfl, index_div0};
   //]

   //[ generated due to goto in signal handler
   static void * labels[]={
		&&normalStart, 
		&&label_exit,
		&&label_restart};
   enum signalGoto{
		noSignal1=0,
		index_exit, 
		index_restart};
   int sigGoto = 0;

restart:
   goto *labels[sigGoto];
   //]

normalStart:    
   printf("------------------\n");

   //[ autogenerated due to ON statements in PROC
   try {
   //]

      _k = pearlrt::Fixed<15>(2);

label_restart:

      //[ generated from ON
      sigActions[index_arith].setActionIndex(1);
      //]
       
      if (p == pearlrt::Fixed<15>(1)) {
         //[ generated from ON
         sigActions[index_overfl].setActionIndex(2);
      //]
      }

      if (p > pearlrt::Fixed<15>(5)) {
         //[ generated from ON
         sigActions[index_overfl].setActionIndex(3);
         //]
      }
     
      //[ generated from ON
      sigActions[index_div0].setActionIndex(4);
      //]
      if (p == pearlrt::Fixed<15>(11)) {
         _k = pearlrt::Fixed<15>(10)/(_k-_k);
      }

      {
         pearlrt::Fixed<15> _i;    // loop counter is auto gen.
         for (_i = 1;
            _i<=pearlrt::Fixed<15>(100); 
            _i = _i + pearlrt::Fixed<15>(1)) {
             _k =  _k * _k;
         }
      }
label_exit:
	/* nop */ ;
   //[ generated due to ON statements in PROC
   } catch (pearlrt::Signal & s) {
      int index = pearlrt::ScheduleSignalAction::getAction(
		&s, NBROFSIGACTIONS, sigActions);
      switch(index) {
          default: throw;   // this signal wasnot scheduled
          //[ generated from signal action #1 in PROC
          case 1:
             printf("proc x: arithmetic error (returning)\n");
             return; // if in PROC; else me->terminate(me);
          //]
          //[ generated from signal action #2 in PROC
          case 2:
             printf("proc x: Overflow occured (terminating)\n");
             me->terminate();
          //]
          //[ generated from signal action #3 in PROC
          case 3: 		 // signal action #3
              try {
                 // code for PUT 'PROC X: Overflow occured'
                 // TO console BY A, SKIP;
                 printf("proc x: Overflow occured (returning)\n");
               } catch (pearlrt::Signal &s) {
//                  if (!console.updateRST(s)) {
//                     throw;
//                  }
               }
               return; // if in PROC; else me->terminate(me);
          //]
          //[ generated from signal action #4 in PROC
	  case 4:               // signal action #4
             try {
                // code for PUT 'PROC X: divide by 0 (restarting)'
                // TO console BY A, SKIP;
                 printf("proc x: divide by 0 (restarting)\n");
              } catch (pearlrt::Signal &s) {
 //                if (!console.updateRST(s)) {
//                    throw;
//                 }
              }
              if (p == pearlrt::Fixed<15>(6)) {
	         sigGoto = index_exit;
                 goto restart;                 
              } else {
                 p = pearlrt::Fixed<15>(6);
	         sigGoto = index_restart;
                 goto restart;                 
              }
          //]
      } // end of switch
   } // end of catch
}
\end{CppCode}


Remarks:
\begin{description}
\item[generated identifiers]  may use different decorations if
   suitable for the compiler.
\item[sigActions] must be of type auto to provide task specific 
   signal reactions.
\item[sigGoto] must be of type auto to provide task specific 
   signal reactions.
\item[labels] should be defined as static, for speedup reasons
\end{description}

\section{C++ Code for Signal Handler with RST-Tag}
When a RST-element is specified in the ON-statement, the current
RST-value must be read into the given variable.

\begin{PEARLCode}
DCL Errorcode FIXED(15);
...
   ! 2nd ON statement in proc/task
   ON overfl RST(Errorcode) ...
\end{PEARLCode}

Currently there is a problem with the headers of Signal and Fixed.
Therefore the type \verb|Fixed<x>| may not be used inside the Signal-class.
The statement {\em RST(Errorcode)} modifies the code in the signal handler
alternative towards.

\begin{CppCode}
Fixed<15> _Errorcode;
...
case 2:
   _Errorcode.x = s.getCurrentRst();
   ...
\end{CppCode}
Obviously the used variable in RST must be defined on module,
 task or procedure level. 

\section{Induce}
The statement {\em INDUCE} simulates a signal and causes the trigger 
of the scheduled reaction.
Induce may specify a RST-value, which causes to raising of the 
signal with the given number.
In this case a temporary RST-value is set inside the signal structure.
This does not affect the identification of the signal.


\begin{PEARLCode}
INDUCE div_0;
INDUCE div_0 RST (1234);
\end{PEARLCode}

\begin{CppCode}
generalized_div0->induce();
generalized_div0->induce(1234);
\end{CppCode}

\begin{PEARLCode}
SYSTEM;
   TaskRunningSignal: Signal(102);
   Arithmetic: Signal(200);

PROBLEM;
   SPC TaskRunningSignal SIGNAL;
   SPC Arithmetic SIGNAL;
   DCL Errorcode Fixed;
   DCL counter FIXED INIT(0);

Prog: TASK MAIN;
    ALL 1 SEC ACTIVATE Demo;
END;

Demo: TASK; 
   counter := counter + 1;

   ON TaskRunningSignal RST(ErrorCode)
	PUT 'TaskSignal occured with RST=', ErrorCode TO console BY A,F,SKIP;
   ON Arithimetic RST(ErrorCode)
	PUT 'Arithmetic problem occured with RST=', ErrorCode TO console BY A,F,SKIP;

   Test;
END;

Test: PROC;
   CASE counter
   ALT /* 1 */ INDUCE TaskRunningSignal RST(200);
   ALT /* 2 */ INDUCE TaskRunningSignal;
   ALT /* 3 */ INDUCE TaskRunningSignal RST(200);
   FIN;
END;
\end{PEARLCode}

This program will produce the output
\begin{verbatim}
TaskSignal occured with RST=200
TaskSignal occured with RST=102
TaskSignal occured with RST=200
\end{verbatim}
The setting of the RST-value to 102 (which is the error code of 
the {\em Arithmetic} does not affect the signal identification.
The setting of the RST-value is done with another version of the 
induce()-method, which allows an {\em int} value as parameter.

Since signals are global objects the feature of setting the RST at
induce should be used with care. The implementation is not safe against
race conditions.

Sample C++ code for procedure {\em Test}:
\begin{CppCode}
void _Test(pearlrt::Task * me) {
   switch(_counter.x) {
      case 1: 
       generalized_TaskRunningSignal->induce(200);
       break;
      case 2: 
       generalized_TaskRunningSignal->induce();
       break;
      case 3: 
       generalized_TaskRunningSignal->induce(200);'
       break;
    }
}
\end{CppCode}

