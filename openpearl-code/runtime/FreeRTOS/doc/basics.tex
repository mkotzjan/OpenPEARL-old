\section{System Overview}
FreeRTOS is used as scheduler. This system was selected, because
\begin{itemize}
\item it provides preemtive priority scheduling
\item it is avaliable on nearly every micro controller system
\item it is available in source code
\end{itemize}

PEARL tasks are mapped on FreeRTOS tasks. The required macros
are provided by \texttt{GenericTask.h}.

The FreeRTOS system has the following disadvantages:
\begin{enumerate}
\item task control blocks and task stack are allocated on the system
   heap on task creation. This may lead to memory fragmentation
\item ... and more ?? ... 
\end{enumerate}


\section{System Architecture}
There are some problems with the linker and \texttt{weak} symbols.
The effect is that \texttt{weak} symbols are not overwritte, if
a corresponding strong symbol resides in a library.
\texttt{weak} symbols are used in the interrupt vector table. This
leads to
\begin{enumerate}
\item an object file which contains the interrupt service handlers like
   reset-handler, timer and other device service routines
\item a runline library with the remaining object files
\end{enumerate}

\subsection{System Initialization}
The system initialization is described in the target specific part.

\subsection{Linkage}
The gcc linker option LTO is used to remove unused functions and data. 
The individual test and application programs are linked together with 
the static library.

\section{Unit Tests}
The linux trunk of the project, which is the main trunk, uses the
goole test framework to run unit tests on nearly all internal classes.
The complete gtest framework is too large to be run on the microcontroller.

A simple test framework, which supports the same syntax as gtest is provided
to run the unit tests on the target system.
The implementation resides in the files\footnote{ugtest stands for $\mu$gtest}

 \verb|ugtests/simpleTests.h| and 
 \verb|ugtests/simpleTests.cc| 

The following elements of gtest are provided:
\begin{description}
\item[TEST] defines a unit test
\item[EXPECT\_EQ] tests, whether the two given parameters are equal, 
   assuming an operator== exists.
\item[EXPECT\_STREQ] tests, whether the two given parameters are equal c string 
\item[EXPECT\_TRUE] tests, whether the  given parameter is non zero. 
\item[EXPECT\_FALSE] tests, whether the  given parameter is zero. 
\item[EXPECT\_THROW] tests, whether the first given expression throws
   an exception of the type as given as  second parameter.
\item[ASSERT\_...] is defined for all listed EXPECT-versions. 
\end{description}

\section{Time Base}
The presence of a real time clock (RTC) is not guaranteed on a
microprocessor system.
This is no big problem for most PEARL applications, since they works 
in many cases with relative times.

FreeRTOS provides a tick based system time. The resolution is specified
in the file \texttt{FreeRTOSConfig.h} --- 
usually set in OpenPEARL to $1 ms$.

Some applications many require absolute times and more precise timing.
Thats the reason why the FreeRTOS provided timers are not used. 
There is an implemtation of the \texttt{itimer} system. This 
implementation allows the specification of a time source.
This system was introduced  by Jonas Meyer in his thesis.

\subsection{class FreeRTOSClock}
The class TaskTimer provides the FreeRTOS tick as
base for itimers.
The absolute time is set to 1.1.2016 0:0:0 at system startup.

Other time sources are plattform specific and described there.

\subsection{Added Features to FreeRTOS and glib-c}
There are some additional functions in \texttt{FreeRTOS/addOns},
 which supply the toolchain
with UNIX-like time functions like gettimeofday.
\begin{description}
\item[clock.c]provides gettimeofday and some more
\item[timer.c] provides the itimers
\end{description}

Implementation of FreeRTOS Hooks:
\begin{description}
\item[assert.c] provides the implementation of the function
  \texttt{assert()}, which is used in FreeRTOS to verify
  proper operation
\item[FreeRTOSHooks.c] provides some error hooks for stack and head overflow.
\end{description}

\section{Task Mapping}

The task mapping is described in the thesis of
 Florian Mahlecke in detail.

In short:
\begin{itemize}
\item GenericTask.h defines C-macros for task declararion and specification
\item Each PEARL task is derived from the class {\em Task} (\verb|Task.cc|).
      The tasking methods are implemented in \verb|Task.cc|.
\item Each PEARL task implements the method \verb|task(TaskCommon*me)| 
    with the C++ code generated by the compiler.
\item {
  \begin{description}
  \item[class Task] provides the required plattform specific implementation
     of the tasking methods
  \item[class TaskTimer] provides the time related stuff for the tasking
     methods. This class is working on base of the itimer extension 
     in \texttt{FreeRTOS/addOns}
  \item[class PrioMapper] provides a 1:1 mapping of PEARL to FreeRTOS
      priorities
  \end{description}
}
\end{itemize}



\subsection{To be done}
  \begin{itemize}
  \item allocate static TCB+Stack as patch for FreeRTOS to avoid 
    memory fragementation due to ACTIVATE/TERMINATE cycles.
  \item provide aute restart at task termination
  \item provide device drivers for the standard devices (UART, SD, Ethernet)

  \end{itemize}


