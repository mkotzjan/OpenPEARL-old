\section{Cortex-M Interrupt Priorities}

\subsection{Cortex-M Basics}

As descibed in \cite{yiu}, the Cortex-M processor supports
 in principle up to 256 interrupt priorities.
Lower numbers denote better priorities.
The execution of interrupt service routines is carried out in order of
the priority of the interrupt.
There is also a grouping of interrupt priorities which defines, that 
interrupts of the same group are executied sequentially in order of the
priority.

The cmsis library in LPCOpen defines \verb|#define NVIC_PRIO_BITS 5|

This means, that only the upper 5 bits of the priority level is regarded.

\begin{table}
\begin{tabular}{|c|c|c|c|c|c|c|c|}
\hline
Bit 7 & Bit 6 & Bit 5 & Bit 4 & Bit 3 & Bit 2 & Bit 1 & Bit 0 \\
\hline
\multicolumn{5}{|c|}{Implemented}&\multicolumn{3}{c|}{Not implemented} \\
\multicolumn{5}{|c|}{}           &\multicolumn{3}{c|}{read as 0} \\
\hline
\multicolumn{7}{|l|}{Preemption Priorities (Group=0)}& x \\
\hline
\multicolumn{4}{|l|}{Preemption Priorities (Group=3)}& sub & x & x & x \\
\hline
\end{tabular}
\caption{Priority Level register with 5-bit implemented. With 
priority grouping of 0, there are no subgroups due to 5 implemented
interrupt priority bits.\cite{yiu}}
\end{table}

The cmsis-function  \verb|NVIC_SetPriority| shifts the given priority value 
according the number of implemented priority bits. This leads to
valid parameters from 0 to 31 --- reflecting the processor interrupts
0,8,16, ...248.

To set the interrupt grouping, the cmsis function 
\verb|NVIC_SetPriorityGrouping| is supplied.
The priority group (which is a plain number) defines the upmost bit of the
interrupt priority as the most significant bit
 of the socalled subpriority field.
In reset state, the priority group is 0 resulting in preemptive interrupts
 with numbers in bit[7:1] and a sub priority field bit[0].
Since only the implementation of the LPC1768 supports only 5 priority bits, we
get a sub priority field, which is always 0 and a 32 preemptive 
interrupt levels. Interrupt priority grouping become effectiv with a group 3.

\subsection{FreeRTOS Settings}
FreeRTOS states
\begin{itemize}
\item  that there may be some grouping on interrupt priorities 
by third party software. 
\item the priority value in FreeRTOS are already left adjusted according the
number in priority bits. 
\end{itemize}
FreeRTOS defines the priority of the kernel to be the highest number, which
means the lowest interrupt priority. In our case it is 31, which results into
248. 
FreeRTOS reserves some priorities, which are dedicated to interrupts, which do not interact with FreeRTOS. They are never blocked by FreeRTOS.
This level is defined as 5, resulting in priority 40.

Interrupts from device drivers may use the interrupt levels from 5 to 30.
All of them are preemptive. 
The preemption of interrupts may leed to a stack problem, it is a part of
system optimization to use different interrupt levels in the future.
As long as interrupt preemption is not needed, all device drivers should use 
a common interrupt priority of 10, which leaves space in both directions.
Since there is no subpriority avaliable, the 
interrupts are executed according the interrupt number --- 
the smaller exception number is executed first.


